% Educational Problem Manager (EPM) Maintenance Manual.
%
% File:         epm_maintenance.tex
% Author:       Bob Walton (walton@acm.org)
% Date:		See \date below.

% The authors have placed EPM (its files and the
% content of these files) in the public domain; they
% make no warranty and accept no liability for EPM.
  
\documentclass[12pt]{article}

\usepackage[T1]{fontenc}
\usepackage{lmodern}
\usepackage{makeidx}

\makeindex

\setlength{\oddsidemargin}{0in}
\setlength{\evensidemargin}{0in}
\setlength{\textwidth}{6.5in}
\setlength{\textheight}{8.5in}
\raggedbottom

\setlength{\unitlength}{1in}

\pagestyle{headings}
\setlength{\parindent}{0.0in}
\setlength{\parskip}{1ex}

\setcounter{secnumdepth}{5}
\setcounter{tocdepth}{5}
\newcommand{\subsubsubsection}[1]{\paragraph[#1]{#1.}}
\newcommand{\subsubsubsubsection}[1]{\subparagraph[#1]{#1.}}

% Begin \tableofcontents surgery.

\newcount\AtCatcode
\AtCatcode=\catcode`@
\catcode `@=11	% @ is now a letter

\renewcommand{\contentsname}{}
\renewcommand\l@section{\@dottedtocline{1}{0.1em}{1.4em}}
\renewcommand\l@table{\@dottedtocline{1}{0.1em}{1.4em}}
\renewcommand\tableofcontents{%
    \begin{list}{}%
	     {\setlength{\itemsep}{0in}%
	      \setlength{\topsep}{0in}%
	      \setlength{\parsep}{1ex}%
	      \setlength{\labelwidth}{0in}%
	      \setlength{\baselineskip}{1.5ex}%
	      \setlength{\leftmargin}{0.8in}%
	      \setlength{\rightmargin}{0.8in}}%
    \item\@starttoc{toc}%
    \end{list}}
\renewcommand\listoftables{%
    \begin{list}{}%
	     {\setlength{\itemsep}{0in}%
	      \setlength{\topsep}{0in}%
	      \setlength{\parsep}{1ex}%
	      \setlength{\labelwidth}{0in}%
	      \setlength{\baselineskip}{1.5ex}%
	      \setlength{\leftmargin}{1.0in}%
	      \setlength{\rightmargin}{1.0in}%
	      }%
    \item\@starttoc{lot}%
    \end{list}}

\catcode `@=\AtCatcode	% @ is now restored

% End \tableofcontents surgery.

\newenvironment{indpar}[1][0.4in]%
	{\begin{list}{}%
		     {\setlength{\itemsep}{0in}%
		      \setlength{\topsep}{0in}%
		      \setlength{\parsep}{1ex}%
		      \setlength{\labelwidth}{#1}%
		      \setlength{\leftmargin}{#1}%
		      \addtolength{\leftmargin}{\labelsep}}%
	 \item}%
	{\end{list}}

\newcommand{\TT}[1]{{\tt \bfseries #1}}
\newcommand{\key}[1]{{\bf \em #1}}
\newcommand{\EOL}{\penalty \exhyphenpenalty}
\newcommand{\pagref}[1]{p\pageref{#1}}
\newcommand{\FSTACK}[2]{{\tt \bfseries
    #1\begin{tabular}[t]{@{}l@{}}#2\end{tabular}}}
\newcommand{\PSTACK}[1]{\begin{tabular}[t]{@{}l@{}}
                        #1\end{tabular}}

\newcommand{\ITEM}{\hspace*{-0.2in}}
\newcommand{\TTITEM}[1]{\hspace*{-0.2in}{\TT{#1}}\\}
\newcommand{\BFITEM}[1]{\hspace*{-0.2in}{{\bf #1}}\\}

\begin{document}
        
\title{Educational Problem Manager\\
Maintenance Manual}

\author{Robert L. Walton}

\date{May 19, 2020}
 
\maketitle

\begin{center}
{\large \bf Notice}
\\[2ex]
\begin{minipage}{5.5in}
The authors have placed EPM (its files and the content of these files) in
the public domain; they make no warranty and accept
no liability for EPM.
\end{minipage}
\end{center}
\begin{center}
\large \bf Table of Contents
\end{center}

\bigskip

\tableofcontents 

\newpage

\section{Introduction}

See the Administration Manual for instructions on how to install
and run an EPM web site.
This document is a manual for EPM system maintainers.

\section{Names}

\section{Data Files}

\begin{center}
\begin{tabular}{|l|l|l|l|l|l|}
\hline
Name & Format & Description & Creators & Updaters & Readers \\
\hline\hline
\TT{admin} & dir & \pagref{ADMIN}
	& login & user & \\ 
\hline
\TT{admin/browser} & dir & \pagref{ADMIN/BROWSER}
	& login & user & \\ 
\hline
\TT{admin/browser/BID} & bid & \pagref{ADMIN/BROWSER/BID}
	& login & & \\ 
\hline
\TT{admin/email/EMAIL} & email & \pagref{ADMIN/EMAIL/EMAIL}
	& user & & \\ 
\hline
\TT{admin/email} & dir & \pagref{ADMIN/EMAIL}
	& login & user & \\ 
\hline
\TT{admin/submit} & dir & \pagref{ADMIN/SUBMIT}
	& run & run & \\ 
\hline
\TT{admin/users} & dir & \pagref{ADMIN/USERS}
	& login & user & \\ 
\hline
\TT{admin/users/UID} & dir & \pagref{ADMIN/USERS/UID}
	& user & & \\ 
\hline
\FSTACK{admin/}{users/UID/\\UID.info} & json
	& \pagref{ADMIN/USERS/UID/UID.INFO}
	& user & & \\ 
\hline
\TT{users} & dir & \pagref{USERS} & login & user & \\ 
\hline
\TT{users/UID} & dir & \pagref{USERS/UID}
	& user & \PSTACK{problem\\project} & \\ 
\hline
\TT{users/UID/+lists+} & dir & \pagref{USERS/UID/+LISTS+}
	& user & \PSTACK{list\\favorites} & project \\ 
\hline
\TT{users/UID/PROBLEM} & dir & \pagref{USERS/UID/PROBLEM}
	& \PSTACK{problem\\project} & & \\ 
\hline
\FSTACK{users/}{UID/PROBLEM/\\PROBLEM.cc}
    & C++ & \pagref{USERS/UID/PROBLEM/PROBLEM.cc}
    	& \PSTACK{problem\\(upload)} & & \\ 
\hline
\TT{projects} & dir & \pagref{PROJECTS} & login & maint & \\ 
\hline
\TT{solutions} & dir & \pagref{SOLUTIONS} & login & maint & \\ 
\hline
\end{tabular}
\end{center}

\section{Web Pages}

\subsection{Login}

\begin{center}
{\bf Login Page Files}
\\[1ex]
\begin{tabular}{lllll}
\TT{admim/browser/BID}	& create  & update & read \\
\TT{admim/email/EMAIL}	& -       & update & read \\
\TT{admim/login.log}	& create  & update & -    \\
\end{tabular}
\\[3ex]
{\bf Login Page Session Data}
\\[1ex]
\begin{tabular}{lllll}
\TT{EPM\_EMAIL}	& -       & -      & read \\
\TT{EPM\_UID}	& create  & -      & -    \\
\TT{EPM\_IPADDR}& -       & -      & read \\
\TT{EPM\_SESSION\_TIME}
                & -       & -      & read \\
\end{tabular}
\end{center}

\subsection{User}

\begin{center}
{\bf User Page Files}
\\[1ex]
\begin{tabular}{lllll}
\TT{admim/browser/BID}	& create  & update & read \\
\TT{admim/email/EMAIL}	& -       & update & read \\
\TT{admim/login.log}	& create  & update & -    \\
\end{tabular}
\\[3ex]
{\bf User Page Session Data}
\\[1ex]
\begin{tabular}{lllll}
\TT{EPM\_EMAIL}	& -       & -      & read \\
\TT{EPM\_UID}	& create  & -      & read \\
\TT{EPM\_IPADDR}& -       & -      & read \\
\TT{EPM\_SESSION\_TIME}
                & -       & -      & read \\
\end{tabular}
\end{center}

\subsection{Problem}

\begin{center}
{\bf Problem Page Files}
\\[1ex]
\begin{tabular}{lllll}
\TT{users/UID/PROBLEM}	                & create  & update & read & delete
\\[1ex]
\TT{users/UID/PROBLEM/PROBLEM.tex}	& upload  & -      & read & delete \\
\TT{users/UID/PROBLEM/PROBLEM.pdf}	& create  & -      & read & delete
\\[1ex]
\TT{users/UID/PROBLEM/PROBLEM.c}	& upload  & -      & read & delete \\
\TT{users/UID/PROBLEM/PROBLEM.cc}	& upload  & -      & read & delete \\
\TT{users/UID/PROBLEM/PROBLEM.java}	& upload  & -      & read & delete \\
\TT{users/UID/PROBLEM/PROBLEM.py}	& upload  & -      & read & delete \\
\TT{users/UID/PROBLEM/PROBLEM}		& create  & -      & read & delete \\
\TT{users/UID/PROBLEM/PROBLEM.class}	& create  & -      & read & delete \\
\TT{users/UID/PROBLEM/PROBLEM.pyc}	& create  & -      & read & delete
\\[1ex]
\TT{users/UID/PROBLEM/XXXX-PROBLEM.c}	& upload  & -      & read & delete \\
\TT{users/UID/PROBLEM/XXXX-PROBLEM.cc}	& upload  & -      & read & delete \\
\TT{users/UID/PROBLEM/XXXX-PROBLEM.java}& upload  & -      & read & delete \\
\TT{users/UID/PROBLEM/XXXX-PROBLEM.py}	& upload  & -      & read & delete \\
\TT{users/UID/PROBLEM/XXXX-PROBLEM}	& create  & -      & read & delete \\
\TT{users/UID/PROBLEM/XXXX-PROBLEM.class}
					& create  & -      & read & delete \\
\TT{users/UID/PROBLEM/XXXX-PROBLEM.pyc}	& create  & -      & read & delete
\\[1ex]
\TT{users/UID/PROBLEM/XXXX-PROBLEM.in}	& upload  & -      & read & delete \\
\TT{users/UID/PROBLEM/XXXX-PROBLEM.sin}	& create  & -      & read & delete \\
\TT{users/UID/PROBLEM/XXXX-PROBLEM.sout}& create  & -      & read & delete \\
\TT{users/UID/PROBLEM/XXXX-PROBLEM.fout}& create  & -      & read & delete \\
\TT{users/UID/PROBLEM/XXXX-PROBLEM.ftest}& create  & -      & read & delete \\
\TT{users/UID/PROBLEM/XXXX-PROBLEM.dout}& create  & -      & read & delete \\
\TT{users/UID/PROBLEM/XXXX-PROBLEM.score}& create  & -      & read & delete \\
\end{tabular}
\\[3ex]
{\bf Problem Page Session Data}
\\[1ex]
\begin{tabular}{lllll}
\TT{EPM\_EMAIL}	& -       & -      & read \\
\TT{EPM\_UID}	& -       & -      & read \\
\TT{EPM\_PROBLEM}
		& create  & update & read \\
\end{tabular}
\end{center}

\section{Overview}

Here we list adminstrative files and transactions, and give a
brief description of each.

\subsection{Administrative Files}

EPM uses administrative files to direct generation of files from
other files, to keep track of visibility
permissions, and for other things.  All these files are in JSON
format.  EPM does \underline{not} use any data base (like MYSQL).

Administrative files are \underline{not} uploadable by the user.
They are made by commands issued by the user to web pages, e.g.,
the \TT{PPPP.score} file is made by the user commanding
the \TT{PPPP.run} file, and the user can make their own
\TT{\ldots-PPPP.run} file using a web page.  When administrative
commands are made using a web page, the code associated with the
page checks that the file being made does not violate security.

The following administrative files the user will encounter.
Unless otherwise noted, these are visible to the user.

\begin{indpar}
\TTITEM{XXXX.mk}  File specifying how to make the file \TT{XXXX},
when this last file is generated from other files.  E.g.,
the \TT{UUUU/PPPP/00-000-PPPP.out.mk} file tells how to make
the \TT{UUUU/PPPP/00-000-PPPP.out} from 
the \TT{system/PPPP/00-000-PPPP.in},
the \TT{system\EOL /\EOL PPPP/\EOL generate-PPPP},
and \TT{UUUU/PPPP/PPPP}.

A \TT{XXXX.mk} file also tells how to make itself, so there is
no need for \TT{XXXX.mk.mk} files.  \TT{XXXX.mk} files are
made from template files.
\\[1ex]
\TTITEM{system/template/X.E->Y.F.tpl}
Template file used to make \TT{.mk} or other administrative
files.  A typical template file name ends with \TT{X.cc->X.tmpl}
and is used to make a \TT{X.mk} file that specifies how to
make a \TT{X} file from a \TT{X.cc} file, where \TT{X} is a
parameter to the template file.  In general single upper case
letters are used as parameter names for template files, and
may also appear in the names of the template files themselves.

Template files are located and used by code in web pages that
make administrative files.
\\[1ex]
\TTITEM{UUUU/credentials} Credential file for user.  Specifies
user email addresses, ip addresses, dates ip addresses last
certified and last used.
\\[1ex]
\TTITEM{UUUU/logins} Login history of user.
\\[1ex]
\TTITEM{UUUU/PPPP/uploads} Upload history of user for problem \TT{PPPP}.
Note that a user's uploads for a problem are automatically
checked into a per-user, per-problem \TT{git} database
which can be cloned by the user and partially inspected
(in particular to obtain difference listings) using EPM
web pages.
\\[1ex]
\TTITEM{UUUU/PPPP/runs} Run history of user for problem \TT{PPPP}.
\\[1ex]
\TTITEM{DDDD/user.perm} Permission control file for arbitrary
users for all files in directory \TT{DDDD} (e.g., in \TT{system},
\TT{system/PPPP}) and its subdirectories.
\\[1ex]
\TTITEM{DDDD/UUUU.perm} Permission control file for
user \TT{UUUU} and files in directory \TT{DDDD} (e.g., in directory
\TT{UUUU}).
\end{indpar}

In the above, \TT{system/} is actually a project directory.
There can be many project, and the \TT{UUUU/PPPP/UUUU.perm} file
can point to any of them, in place of \TT{system/}.

\subsection{Transactions}

The only way a user can interact with EPM is via transactions.
Each transaction is executed by entering small amounts of text
on a web page and clicking appropriately.

The common transations are:

\begin{indpar}
\BFITEM{download}  Download any file visible to the user.
\\[1ex]
\BFITEM{inspect}  Inspect various visible JSON administrative files
in a more readable format.
\\[1ex]
\BFITEM{directory}  Create or destroy problem directories and designate
a current problem directory.
\\[1ex]
\BFITEM{upload}  Upload any program source (\TT{.c}, \TT{.cc}, \TT{.java},
\TT{.py}, or \TT{.lsp}) file or any program input (\TT{.in}) file into
the current problem directory.
\\[1ex]
\BFITEM{make}  Create, edit, and destroy \TT{.mk} files, whose existence
causes the associated derived files to be made upon demand, in the
current problem directory.
\\[1ex]
\BFITEM{run}  Create, edit, and destroy \TT{.run} files, which define
runs that cause batches of derived files to be created, in the
currrent problem directory.  Execute
designated runs.
\\[1ex]
\BFITEM{permission}  Create, edit, and destroy \TT{.perm} files that control
permissions and visibility.
\\[1ex]
\BFITEM{credentials}  Inspect and remove credentials of the current user.
\\[1ex]
\BFITEM{project}  Create or destroy project directories and designate
project directories that are visible to the current problem directory.
\\[1ex]
\BFITEM{move}  Move files between the current problem directory and a
project directory.
\end{indpar}

Program source files do not have to be problem solution files.
Any program can be run so long as it opens no files and does
all its input/output via file descriptors.  Program output
can be put into \TT{.out}, \TT{.fout}, \TT{.debug}, \TT{.info}, or
\TT{.disp} files.  The last kind of file encodes X-windows commands
that can be displayed in an X-window or pdf window or placed in a
\TT{.pdf} file.  Programs an also interact with terminal windows;
for example, a program can be written to calculate combinations
(i.e., N choose K).

Problem solution programs are run without arguments, unless they
are begin debugged, in which case their output is put into \TT{.debug}
files.  Other programs can be run with or without arguments.



\end{document}
