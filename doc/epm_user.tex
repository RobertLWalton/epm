% Educational Problem Manager (EPM) User Manual
%
% File:         epm.tex
% Author:       Bob Walton (walton@acm.org)
% Date:		See \date below.
  
\documentclass[12pt]{article}

\usepackage[T1]{fontenc}
\usepackage{lmodern}
\usepackage{makeidx}

\makeindex

\setlength{\oddsidemargin}{0in}
\setlength{\evensidemargin}{0in}
\setlength{\textwidth}{6.5in}
\setlength{\textheight}{8.5in}
\raggedbottom

\setlength{\unitlength}{1in}

\pagestyle{headings}
\setlength{\parindent}{0.0in}
\setlength{\parskip}{1ex}

\setcounter{secnumdepth}{5}
\setcounter{tocdepth}{5}
\newcommand{\subsubsubsection}[1]{\paragraph[#1]{#1.}}
\newcommand{\subsubsubsubsection}[1]{\subparagraph[#1]{#1.}}

\newcommand{\TT}[1]{{\tt \bfseries #1}}
\newcommand{\EOL}{\penalty \exhyphenpenalty}

% Begin \tableofcontents surgery.

\newcount\AtCatcode
\AtCatcode=\catcode`@
\catcode `@=11	% @ is now a letter

\renewcommand{\contentsname}{}
\renewcommand\l@section{\@dottedtocline{1}{0.1em}{1.4em}}
\renewcommand\l@table{\@dottedtocline{1}{0.1em}{1.4em}}
\renewcommand\tableofcontents{%
    \begin{list}{}%
	     {\setlength{\itemsep}{0in}%
	      \setlength{\topsep}{0in}%
	      \setlength{\parsep}{1ex}%
	      \setlength{\labelwidth}{0in}%
	      \setlength{\baselineskip}{1.5ex}%
	      \setlength{\leftmargin}{0.8in}%
	      \setlength{\rightmargin}{0.8in}}%
    \item\@starttoc{toc}%
    \end{list}}
\renewcommand\listoftables{%
    \begin{list}{}%
	     {\setlength{\itemsep}{0in}%
	      \setlength{\topsep}{0in}%
	      \setlength{\parsep}{1ex}%
	      \setlength{\labelwidth}{0in}%
	      \setlength{\baselineskip}{1.5ex}%
	      \setlength{\leftmargin}{1.0in}%
	      \setlength{\rightmargin}{1.0in}%
	      }%
    \item\@starttoc{lot}%
    \end{list}}

\catcode `@=\AtCatcode	% @ is now restored

% End \tableofcontents surgery.

\newenvironment{indpar}[1][0.4in]%
	{\begin{list}{}%
		     {\setlength{\itemsep}{0in}%
		      \setlength{\topsep}{0in}%
		      \setlength{\parsep}{1ex}%
		      \setlength{\labelwidth}{#1}%
		      \setlength{\leftmargin}{#1}%
		      \addtolength{\leftmargin}{\labelsep}}%
	 \item}%
	{\end{list}}

\newcommand{\ITEM}{\hspace*{-0.2in}}
\newcommand{\TTITEM}[1]{\hspace*{-0.2in}{\TT{#1}}\\}
\newcommand{\BFITEM}[1]{\hspace*{-0.2in}{{\bf #1}}\\}
\newcommand{\key}[1]{{\bf \em #1}}

\begin{document}
        
\title{Educational Problem Manager\\User Manual}

\author{Robert L. Walton}

\date{January 4, 2020}
 
\maketitle

\begin{center}
\large \bf Table of Contents
\end{center}

\bigskip

\tableofcontents 

\newpage

\section{Introduction}

The Educational Problem Manager permits educational programming
problems to be developed and used.  Its users merely need an
editor on their client computer: all compilation, including
text processing (e.g., latex compilation), is done on the
server.  The interface is a web browser.

The main features are:

\begin{itemize}
\item Server Operating System: linux (CentOS or Ubuntu)
\item Documentation Language: latex (pdflatex)
\item System Programming Language: php (version 7.0 or later)
\item User Programming Languages:
\begin{itemize}
\item C (gcc)
\item C++ (g++)
\item JAVA (OpenJDK)
\item COMMONLISP (SBCL)
\end{itemize}

\end{itemize}

This document is a comprehensive user manual.

\section{Overview}

EPM in effect is a file system, where some files are source
files uploaded by users, and many files are automatically
generated from source files and from other generated files.
The typical user does \underline{not} have control over
the precise commands use to generate files from other
files, but can merely upload source files and cause generated
files to be generated.  The user can also display files
and information derived from files.

It is expected that the user will have files and directories
on his own computer that mirror those uploaded to EPM.
Files that you have uploaded to EPM \underline{cannot}
be downloaded to your own computer.  Some generated files,
such as \TT{.pdf} files generated by uploading a latex
\TT{.tex} file, can be downloaded.

\section{Login}

You log in using your \key{email address} as your user name.
When you first log in your account is created.  This
first login requires two extra actions:
\begin{enumerate}
\item A \key{confirmation number} is sent to your email address,
and you must enter this on the login page.
\item After the confirmation number is accepted, you are
directed to an account information editing page, and you
must enter your name, institution, and institution location.
Your account will not be created until after yoy do this.
You may use `\TT{Self}'
as your institution, although EPM is principally intended
for users associated with a college or university (as
a student or teacher), and if this is the case you should
give the college or university.
\end{enumerate}

After your first login, your browser will have a \key{ticket}
in its local store that will allow you to log in by giving
just your email address.  The ticket has an expiration time,
and after expiring, you will have to create a new ticket
by receiving a new confirmation number and entering it into
the browser.

You first ticket \key{expires} in 2 days; your second ticket in 7 days,
and all subsequent tickets in 30 days.

A ticket is valid only for a particular email address, a particular
browser, and a particular EPM server URL.  However if you move
your computer to a new location and use the same browser, the
ticket will still be valid.  There is a button to delete the
current ticket, which will just require you to go through the
confirmation number process again to create a new ticket.  Unless
your ticket somehow becomes corrupted, you need not use this
button.




\section{Creating a Problem}

So let us look at a detailed example involving a user named
\TT{UUUU} and a problem named \TT{PPPP}.  The problem has
a directory that contains files related to the problem,
and this section discusses these files.

The problems of a user are grouped into project.  However
each user has a default project named \TT{SELF}, and if
the user does not create any other projects, the \TT{SELF}
project is hidden from the user as if the notion or projects
did not exist.  In this section we will assume the concept
of projects is hidden in this way.

\begin{indpar}
\TTITEM{system/PPPP/PPPP.tex}  Problem description in latex.
    Not visible to the user.
\\[1ex]
\TTITEM{system/PPPP/PPPP.pdf}  Problem description made from
    \TT{PPPP.tex} by using \TT{pdflatex}.
    Visible to user.
\\[1ex]
\TTITEM{system/PPPP/supplement-PPPP.tex}  Problem supplementary
    documentation in latex.  Documents generate and filter programs:
    see below.  Not visible to the user.
\\[1ex]
\TTITEM{system/PPPP/supplement-PPPP.pdf}  Problem supplementary
    documentation made from
    \TT{PPPP.tex} by using \TT{pdflatex}.
    Visible to user.
\\[1ex]
\ITEM\begin{tabular}[t]{@{}l}
     \TT{system/PPPP/00-000-PPPP.in} \\
     \TT{system/PPPP/00-001-PPPP.in} \\
     \ldots\ldots\ldots\ldots\ldots \\
     \TT{system/PPPP/01-000-PPPP.in} \\
     \TT{system/PPPP/01-001-PPPP.in} \\
     \ldots\ldots\ldots\ldots\ldots \\
     \end{tabular}
     ~~~~
     \begin{tabular}[t]{p{3in}}
     Test case input files.  Usable in runs,
     but not visible to user except for \TT{00-\ldots.in}
     files that are visible sample inputs.
     \end{tabular}
\\[1ex]
\TTITEM{system/PPPP/generate\_PPPP.cc}  Source code for generate
    program that generates actual test case input from \TT{.in} files.
    Not visible to user.
\\[1ex]
\TTITEM{system/PPPP/generate\_PPPP}  Binary of generate
    program made by using \TT{g++} on \TT{gen\-erate\_\EOL PPPP.cc}.
    Visible to user.
\\[1ex]
\TTITEM{system/PPPP/PPPP.cc}  Judge's solution to the problem.
    Accepts actual test case input from the generate program,
    and produces actual test case output (\TT{.test} files below).
    Not visible to user.
\\[1ex]
\TTITEM{system/PPPP/PPPP}  Binary of judge's solution to the problem,
    made from \TT{PPPP.cc} by using \TT{g++}.
    Not visible to user, but can be used during a run to regenerate
    other \TT{system/PPPP} files.
\\[1ex]
\ITEM\begin{tabular}[t]{@{}l}
     \TT{system/PPPP/00-000-PPPP.test} \\
     \TT{system/PPPP/00-001-PPPP.test} \\
     \ldots\ldots\ldots\ldots\ldots \\
     \TT{system/PPPP/01-000-PPPP.test} \\
     \TT{system/PPPP/01-001-PPPP.test} \\
     \ldots\ldots\ldots\ldots\ldots \\
     \end{tabular}
     ~~~~
     \begin{tabular}[t]{p{3in}}
     Judge's test case output files.  Each is made by running
     the corresponding \TT{.in} file first through the generate
     program and then through the judge's solution program.
     Usable in runs but not visible to user except for \TT{00-\ldots.test}
     files that are visible sample judge's outputs.
     \end{tabular}
\\[1ex]
\TTITEM{system/PPPP/PPPP.run}  Standard `run' file which lists
    the test cases to be run in the order they are to be
    run.  For example:
   `{\TT{00-000-PPPP}, \TT{00-001-PPPP}, \ldots}'.
    Visible to the user.
\\[1ex]
\TTITEM{system/PPPP/sample-PPPP.run}  Ditto but only includes sample tests.
\end{indpar}

All compiled programs follow the convention that they cannot
open files, but must use file descriptors, in particular
the standard input and standard output.  If a solution writes
the standard error, it is deemed to have crashed (because
runtime systems write the standard error if they detect
JAVA errors, memory faults, etc.).  Sometimes a non-solution
will read from file descriptor 3, which is used as a secondary
input when two inputs are required, as by a generate program
(or a filter program, described below).

All this is to allow all compiled programs to run in a sandbox,
so as to protect the server from malicious programs, including,
for example, generate programs submitted by users developing
their own problems.

The user directory has the following files, all of which are
visible to the user unless noted otherwise:

\begin{indpar}
\TTITEM{UUUU/PPPP/PPPP.py}  User's solution to the problem.
    May be in a different programming language from the judge's
    solution (e.g., \TT{.py} is python, \TT{.cc} is C++).
    Uploaded by the user (and thus modifiable by the user).
\\[1ex]
\TTITEM{UUUU/PPPP/PPPP}  Binary of users's solution to the problem,
    typically a \TT{bash} script that executes `\TT{python3 PPPP.py}'
    (if the user's solution were C++, \TT{g++} would be used to
    compile \TT{PPPP.cc}).
\\[1ex]
\ITEM\begin{tabular}[t]{@{}l}
     \TT{UUUU/PPPP/00-000-PPPP.out} \\
     \TT{UUUU/PPPP/00-001-PPPP.out} \\
     \ldots\ldots\ldots\ldots\ldots \\
     \TT{UUUU/PPPP/01-000-PPPP.out} \\
     \TT{UUUU/PPPP/01-001-PPPP.out} \\
     \ldots\ldots\ldots\ldots\ldots \\
     \end{tabular}
     ~~~~
     \begin{tabular}[t]{p{3in}}
     User's test case output files.  Each is made by running
     the corresponding \TT{.in} file first through the generate
     program and then through the user's solution program.
     Only sample test case files and the file for the first
     failed test case are visible to the user.
     \end{tabular}
\\[1ex]
\TTITEM{UUUU/PPPP/PPPP.score}  Score associated with
    the `\TT{system/PPPP/PPPP.run}' file.  Either says the run was completely
    correct, or specifies the first test case that failed.
    Also contains a history of past scores.
    Total scores are computed using the number of failed runs before
    the first successful run.
\\[1ex]
\TTITEM{UUUU/PPPP/sample-PPPP.score}  Ditto but only include sample tests
in \TT{system/\EOL PPPP/\EOL sample-\EOL PPPP.run}.
\\[1ex]
\ITEM\begin{tabular}[t]{@{}l}
     \TT{system/PPPP/00-000-PPPP.gin} \\
     \TT{system/PPPP/00-001-PPPP.gin} \\
     \ldots\ldots\ldots\ldots\ldots \\
     \TT{system/PPPP/01-000-PPPP.gin} \\
     \TT{system/PPPP/01-001-PPPP.gin} \\
     \ldots\ldots\ldots\ldots\ldots \\
     \end{tabular}
     ~~~~
     \begin{tabular}[t]{p{3in}}
     Result of running corresponding \TT{.in} files through the 
     generate program.  These files represent the input actually
     seen by the user's \TT{PPPP} solution program.  Not all
     these files are computed: a \TT{.gin} file may be computed
     only if the corresponding \TT{.in} file is visible to the
     user.
     \end{tabular}

\end{indpar}

The user, to develop a problem solution, repeatedly uploads and
the \TT{PPPP.py} file and executes a \TT{.run} file.
Each test produces a \TT{.in},
\TT{.gin}, \TT{.out}, and \TT{.test} file that can be inspected,
along with differences between the \TT{.out} and \TT{.test}
files.

A general principal is that the user is only allowed to look
at input and output files (\TT{.in}, \TT{.gin}, \TT{.test}, \TT{.out} files)
of sample test cases (\TT{00-\ldots} files), and of the `first
failed test case'.  This last is computed by executing a run
using a \TT{.run} file, and is the first test case listed in
that file on which the solution fails.

In the above we have assumed that the problem is such that the
\TT{.out} file is uniquely determined by the \TT{.in} file,
with perhaps minor differences such as spacing and
the exact number of decimal places printed (as long as number
agree within the tolerance specified by the problem).
However some problems have many possible correct solutions:
for example a maze problem asking for a shortest path in which
there may be several shortest pathes.
For these the following files are added.

In the \TT{system/PPPP} directory:

\begin{indpar}
\TTITEM{system/PPPP/filter\_PPPP.cc}  Source code for filter
    program that accepts actual test case input from the generate
    program and actual test case output from a problem solution
    and produces filtered output (\TT{.ftest} or \TT{.fout} files,
    see below).
    Not visible to the user.
\\[1ex]
\TTITEM{system/PPPP/filter\_PPPP}  Binary of filter
    program made by using \TT{g++} on \TT{fil\-ter\_\EOL PPPP.cc}.
    Visible to the user.
\\[1ex]
\ITEM\begin{tabular}[t]{@{}l}
     \TT{system/PPPP/00-000-PPPP.ftest} \\
     \TT{system/PPPP/00-001-PPPP.ftest} \\
     \ldots\ldots\ldots\ldots\ldots \\
     \TT{system/PPPP/01-000-PPPP.ftest} \\
     \TT{system/PPPP/01-001-PPPP.ftest} \\
     \ldots\ldots\ldots\ldots\ldots \\
     \end{tabular}
     ~~~~
     \begin{tabular}[t]{p{3in}}
     Judge's test case filtered output files.  Each is made by running
     the corresponding \TT{.in} file first through the generate
     program, then through the judge's solution program, and lastly
     through the filter program.
     Useable in runs but not visible to user,
     except for \TT{00-\ldots.ftest} files that are visible sample outputs.
     \end{tabular}
\end{indpar}

For test cases on which a solution is successful, the filtered output
typically consists of just one line containing either \TT{OK}, or in
the case of a problem that tries to minimize some number while providing
extra information, just the number.  If the solution is unsuccessful,
the filtered output will have error messages.  For example, if a shortest path
through a maze is to be output, and the \TT{.test} file has the path,
the filtered output will have the length of the path if the path is
legal, and error messages otherwise.

When there is a filter program, the \TT{UUUU/PPPP} directory has:

\begin{indpar}
\ITEM\begin{tabular}[t]{@{}l}
     \TT{system/PPPP/00-000-PPPP.fout} \\
     \TT{system/PPPP/00-001-PPPP.fout} \\
     \ldots\ldots\ldots\ldots\ldots \\
     \TT{system/PPPP/01-000-PPPP.fout} \\
     \TT{system/PPPP/01-001-PPPP.fout} \\
     \ldots\ldots\ldots\ldots\ldots \\
     \end{tabular}
     ~~~~
     \begin{tabular}[t]{p{3in}}
     User's test case filtered output files.  Each is made by running
     the corresponding \TT{.in} file first through the generate
     program, then through the user's solution program, and lastly
     through the filter program.
     Not visible to user except sample test cases and the first failed
     test case.
     \end{tabular}
\end{indpar}

A user may develop a problem.  In this case the \TT{UUUU/PPPP} problem
will have all the files that were in the \TT{system/PPPP} directory
above, all these files will be visible to the user, and the user
will be able to upload all the \TT{.tex}, \TT{.cc}, \TT{.py} etc
files (but not the administrative files in the next section).




\end{document}
