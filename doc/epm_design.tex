% Educational Problem Manager (EPM) Design Manual.
%
% File:         epm_design.tex
% Author:       Bob Walton (walton@acm.org)
% Date:		See \date below.

% The authors have placed EPM (its files and the
% content of these files) in the public domain; they
% make no warranty and accept no liability for EPM.
  
\documentclass[12pt]{article}

\usepackage[T1]{fontenc}
\usepackage{lmodern}
\usepackage{makeidx}

\makeindex

\setlength{\oddsidemargin}{0in}
\setlength{\evensidemargin}{0in}
\setlength{\textwidth}{6.5in}
\setlength{\textheight}{8.5in}
\raggedbottom

\setlength{\unitlength}{1in}

\pagestyle{headings}
\setlength{\parindent}{0.0in}
\setlength{\parskip}{1ex}

\setcounter{secnumdepth}{5}
\setcounter{tocdepth}{5}
\newcommand{\subsubsubsection}[1]{\paragraph[#1]{#1.}}
\newcommand{\subsubsubsubsection}[1]{\subparagraph[#1]{#1.}}

% Begin \tableofcontents surgery.

\newcount\AtCatcode
\AtCatcode=\catcode`@
\catcode `@=11	% @ is now a letter

\renewcommand{\contentsname}{}
\renewcommand\l@section{\@dottedtocline{1}{0.1em}{1.4em}}
\renewcommand\l@table{\@dottedtocline{1}{0.1em}{1.4em}}
\renewcommand\tableofcontents{%
    \begin{list}{}%
	     {\setlength{\itemsep}{0in}%
	      \setlength{\topsep}{0in}%
	      \setlength{\parsep}{1ex}%
	      \setlength{\labelwidth}{0in}%
	      \setlength{\baselineskip}{1.5ex}%
	      \setlength{\leftmargin}{0.8in}%
	      \setlength{\rightmargin}{0.8in}}%
    \item\@starttoc{toc}%
    \end{list}}
\renewcommand\listoftables{%
    \begin{list}{}%
	     {\setlength{\itemsep}{0in}%
	      \setlength{\topsep}{0in}%
	      \setlength{\parsep}{1ex}%
	      \setlength{\labelwidth}{0in}%
	      \setlength{\baselineskip}{1.5ex}%
	      \setlength{\leftmargin}{1.0in}%
	      \setlength{\rightmargin}{1.0in}%
	      }%
    \item\@starttoc{lot}%
    \end{list}}

\catcode `@=\AtCatcode	% @ is now restored

% End \tableofcontents surgery.

\newenvironment{indpar}[1][0.4in]%
	{\begin{list}{}%
		     {\setlength{\itemsep}{0in}%
		      \setlength{\topsep}{0in}%
		      \setlength{\parsep}{1ex}%
		      \setlength{\labelwidth}{#1}%
		      \setlength{\leftmargin}{#1}%
		      \addtolength{\leftmargin}{\labelsep}}%
	 \item}%
	{\end{list}}

\newenvironment{itemlist}[1][0.2in]%
	{\begin{list}{}{\setlength{\labelwidth}{#1}%
		        \setlength{\leftmargin}{\labelwidth}%
		        \addtolength{\leftmargin}{+0.2in}%
		        \addtolength{\linewidth}{-\labelwidth}%
		        \addtolength{\linewidth}{-0.2in}%
		        \renewcommand{\makelabel}[1]{##1\hfill}}
	 \raggedright}%
	{\end{list}}

\newcommand{\TT}[1]{{\tt \bfseries #1}}
\newcommand{\key}[1]{{\bf \em #1}}
\newcommand{\EOL}{\penalty \exhyphenpenalty}
\newcommand{\pagref}[1]{p\pageref{#1}}
\newcommand{\FSTACK}[2]{{\tt \bfseries
    #1\begin{tabular}[t]{@{}l@{}}#2\end{tabular}}}
\newcommand{\STACK}[1]{\begin{tabular}[t]{@{}l@{}}
                        #1\end{tabular}}

\newcommand{\ITEM}{\hspace*{-0.2in}}
\newcommand{\TTITEM}[1]{\hspace*{-0.2in}{\TT{#1}}\\}
\newcommand{\BFITEM}[1]{\hspace*{-0.2in}{{\bf #1}}\\}

\newcommand{\STAR}{{\Large $^\star$}}
\newcommand{\PLUS}[1][]{{$^{+#1}$}}
\newcommand{\QMARK}{{$^{\,\mbox{\footnotesize ?}}$}}

\begin{document}
        
\title{Educational Problem Manager\\
Design Manual}

\author{Robert L. Walton}

\date{July 18, 2020}
 
\maketitle

\begin{center}
{\large \bf Notice}
\\[2ex]
\begin{minipage}{5.5in}
The authors have placed EPM (its files and the content of these files) in
the public domain; they make no warranty and accept
no liability for EPM.
\end{minipage}
\end{center}
\begin{center}
\large \bf Table of Contents
\end{center}

\bigskip

\tableofcontents 

\newpage

\section{Introduction}

This document gives design information for EPM system maintainers.
This document supplements the EPM Help Page documentation for
users and comments in code files for programmers.

\section{Principals}

\subsection{Names}

\begin{enumerate}
\item Names chosen by the user consists of letters, digits,
dash({\tt -}), and underscore({\tt \_}), begin with a letter,
and end with a letter or digit.
See {\tt /include/parameters.php} {\tt \$epm\_name\_re}.
\item Visible file basenames consist of letters, digits, dash({\tt -}), and
underscore({\tt \_}), begin with a letter or digit, and end
with a letter or digit.  Visible extensions, if present, obey the
same rules.
\item Basenames of problem files end with the
      problem name, which may optionally be preceded by a dash({\tt -}) but
      not by any other character.
      See {\tt /include/parameters.php} {\tt \$epm\_filename\_re}.
\item Invisible problem
      file and directory names begin and end with plus({\tt +}).
\item Administrative files may follow other rules.  In particular,
      email addresses have a file whose name is the URL encoded email
      address, and browser tickets have a file whose name is the
      32 hex digit ticket itself.
\end{enumerate}

\subsection{Times}

\begin{enumerate}
\item Times are UTC formatted as per {\tt "\%FT\%T\%z"}.
      See {\tt /include/parameters.php}\\ {\tt \$epm\_time\_format}.
\end{enumerate}

\section{Data Files}

\begin{center}
\begin{tabular}{|l|l|l|l|l|l|}
\hline
Name & Format & Description & Creators & Updaters & Readers \\
\hline\hline
\TT{admin} & dir & \STACK{administrative\\files}
	& login & user & \\ 
\hline
\TT{admin/+lock+} & dir & \STACK{administrative\\lock file}
	& & \STACK{user\\team} & \\ 
\hline
\TT{admin/browser} & dir & \STACK{browser\\tickets}
	& login & user & \\ 
\hline
\TT{admin/browser/TICKET} & 1-line & \pagref{ADMIN/TICKET/TICKET}
	& login & & \\ 
\hline
\TT{admin/email} & dir & email files
	& login & user & \\ 
\hline
\TT{admin/email/EMAIL} & 1-line & \pagref{ADMIN/EMAIL/EMAIL}
	& \STACK{user\\team} & & \\ 
\hline
\TT{admin/users} & dir & \STACK{administrative\\user\\directories}
	& login & user & \\ 
\hline
\TT{admin/users/UID} & dir & \STACK{administrative\\UID user files}
	& user & & \\ 
\hline
\FSTACK{admin/}{users/UID/\\PID.log} & lines & \pagref{ADMIN/USERS/UID/PID.LOG}
	& \STACK{user\\team} & & \\ 
\hline
\FSTACK{admin/}{users/UID/\\PID.inactive} & lines
			& \pagref{ADMIN/USERS/UID/PID.INACTIVE}
	& team & & \\ 
\hline
\FSTACK{admin/}{users/UID/\\UID.info} & json
	& \pagref{ADMIN/USERS/UID/UID.INFO}
	& user & & \\ 
\hline
\TT{users} & dir & \pagref{USERS} & login & user & \\ 
\hline
\TT{users/UID} & dir & \pagref{USERS/UID}
	& user & \STACK{problem\\project} & \\ 
\hline
\TT{users/UID/+lists+} & dir & \pagref{USERS/UID/+LISTS+}
	& user & \STACK{list\\favorites} & project \\ 
\hline
\TT{users/UID/PROBLEM} & dir & \pagref{USERS/UID/PROBLEM}
	& \STACK{problem\\project} & & \\ 
\hline
\FSTACK{users/}{UID/PROBLEM/\\PROBLEM.cc}
    & C++ & \pagref{USERS/UID/PROBLEM/PROBLEM.cc}
    	& \STACK{problem\\(upload)} & & \\ 
\hline
\TT{projects} & dir & \pagref{PROJECTS} & login & maint & \\ 
\hline
\TT{solutions} & dir & \pagref{SOLUTIONS} & login & maint & \\ 
\hline
\end{tabular}
\end{center}

\section{Session Variables}

\begin{center}
\begin{tabular}{|l|l|l|l|l|l|}
\hline
Name & Description & Creators & Updaters & Readers \\
\hline\hline
\TT{EPM\_EMAIL} & \STACK{email used\\for login}
                & \STACK{login\\user} & & all pages \\ 
\hline
\TT{EPM\_UID} & user ID & \STACK{login\\user} & & all pages \\ 
\hline
\TT{EPM\_IPADDR} & \STACK{session\\IP address}
                 & login & & \STACK{login\\user} \\ 
\hline
\TT{EPM\_TIME} & \STACK{session\\time}
                 & login & & \STACK{login\\user} \\ 
\hline
\end{tabular}
\end{center}

\section{Web Pages}

\subsection{Login Page}

\begin{center}
{\bf Login Page Requires}
\\[1ex]
\begin{tabular}{l}
\TT{page/index.php} \\
\TT{include/epm\_random.php} \\
\end{tabular}
\\[3ex]
{\bf Login Page Files}
\\[1ex]
\begin{tabular}{lllll}
\TT{admim/ticket/TICKET}	& create  & -      & read \\
\TT{admim/email/EMAIL}	& -       & update & read \\
\TT{admim/users/UID/login/PID.log}
			& -       & append & - \\
\end{tabular}
\\[3ex]
{\bf Login Page Session Data}
\\[1ex]
\begin{tabular}{lllll}
\TT{EPM\_EMAIL}	& create  & -      & read \\
\TT{EPM\_UID}	& create  & -      & read    \\
\TT{EPM\_IPADDR}& create  & -      & read \\
\TT{EPM\_SESSION\_TIME}
                & create  & -      & read \\
\TT{EPM\_ID\_GEN['+main+']}
                & create  & update & read \\
\TT{EPM\_SESSION}
                & create  & -      & - \\
\end{tabular}
\end{center}

\subsubsection{Login Page File Formats}

\begin{indpar}
\begin{itemlist}
\item[\TT{admin/browser/TICKET} (ticket file):] UID EMAIL
\label{ADMIN/TICKET/TICKET} \\
\begin{tabular}[t]{lp{4.0in}}
TICKET & ticket proper; 32 hexadecimal digit ticket number \\
UID & User ID or '-' if personal and not known \\
EMAIL & Email address \\
\end{tabular}
\\
\begin{itemize}
\item When a person initially logs in to create an account,
the UID is not known when the ticket is created.
\end{itemize}

\item[\TT{admin/email/EMAIL} (email file):] PID ACOUNT ATIME
\label{ADMIN/EMAIL/EMAIL} \\
\begin{tabular}[t]{lp{4.0in}}
EMAIL & Email address encoded with PHP rawurlencode \\
PID & Personal user ID \\
ACOUNT & Number of auto-login periods completed so far. \\
ATIME & Start time of newest (incomplete) auto-login period. \\
\end{tabular}

\item[\TT{admin/users/UID/PID.log} (login log):]~
\label{ADMIN/USERS/UID/PID.LOG} \\
Lines of format:\hspace{0.5in}TIME EMAIL IPADDR BROWSER \\
\begin{tabular}[t]{@{\hspace{0.2in}}lp{3.9in}}
UID & User ID; equals team UID for teams \\
PID & Personal ID; equals UID for personal users \\
EMAIL & Email address used for login \\
TIME & Session time (EPM\_SESSION\_TIME) for login \\
IPADDR & IP address for session, or `\TT{-}' if creation record \\
BROWSER & \STACK{
          \$\_SERVER['HTTP\_USER\_AGENT'] with `(...)'s\\
	  removed and horizontal spaces replaced by `\TT{;}'s;\\
	  or `\TT{-}' if creation record}
\end{tabular}
\\
\begin{itemize}
\item If UID is a team ID, the first line is a creation record
written when UID:PID is created.  No other line is a creation record.
\item This file name and modification time is stored in
EPM\_SESSION and used to abort a session if another session
logs in with the same UID:PID.
\end{itemize}

\item[\TT{admin/users/UID/PID.inactive}:]~
\label{ADMIN/USERS/UID/PID.INACTIVE} \\
Inactive login log file, made by renaming active file
when PID is no longer a member
of UID team.  May be reactivated.


\end{itemlist}
\end{indpar}

\subsubsection{Login Page Transactions}

\begin{enumerate}
\item If admin/emails/EMAIL exists for EMAIL provided by
      user to browser, log existing
      user in and go to Project Page.
\item If admin/emails/EMAIL does NOT exist for EMAIL
      provided by user to browser, give
      the browser a valid ticket and go to User Page.
\end{enumerate}


\subsection{User}

\begin{center}
{\bf User Page Files}
\\[1ex]
\begin{tabular}{lllll}
\TT{admim/browser/BID}	& create  & update & read \\
\TT{admim/email/EMAIL}	& -       & update & read \\
\TT{admim/login.log}	& create  & update & -    \\
\end{tabular}
\\[3ex]
{\bf User Page Session Data}
\\[1ex]
\begin{tabular}{lllll}
\TT{EPM\_EMAIL}	& -       & -      & read \\
\TT{EPM\_UID}	& create  & -      & read \\
\TT{EPM\_IPADDR}& -       & -      & read \\
\TT{EPM\_SESSION\_TIME}
                & -       & -      & read \\
\end{tabular}
\end{center}

\subsubsection{User Page File Formats}

\begin{indpar}
\begin{itemlist}
\item[\TT{admin/users/UID/UID.info} (user info file):]~
\label{ADMIN/USERS/UID/UID.INFO} \\
JSON file with the following components:
\begin{tabular}[t]{ll}
\TT{'uid'} & UID (PID for person, team UID for team) \\
\TT{'sponsor'} & PID; missing if not team \\
\TT{'manager'} & PID; missing if not team \\
\TT{'emails'} & \TT{[} EMAIL \{ \TT{,} EMAIL \}\STAR{} \TT{]};
                missing if team \\
\TT{'members'} & \TT{[} MID \{ \TT{,} MID \}\STAR{} \TT{]};
                missing if not team \\
\TT{'full\_name'} & TEXT \\
\TT{'organization'} & TEXT \\
\TT{'location'} & TEXT \\
\end{tabular}
\\
where
\\
\begin{tabular}[t]{lp{4.0in}}
MID & member ID; PID if available or EMAIL not yet assigned to a user
      personal account otherwise (EMAIL has `\TT{@}' and PID does not) \\
TEXT & plain text \\
\end{tabular}
\\
\begin{itemize}
\item When a team UID.info file is created, MIDs are specified
as EMAILs which are resolved if possible to PIDs.
\item When a person initially creates an account, all
UID.info files are searched and if any have MIDs matching
the new account EMAIL, they are resolved to PIDs.
\end{itemize}

\end{itemlist}
\end{indpar}

\subsubsection{User Page Transactions}

\begin{enumerate}
\item If EPM\_UID not set, get data for new user and create
      new user account if data acceptable.
\item If EPM\_UID exists for a personal account, display
      user data and allow it to be edited.
\item If EPM\_UID exists for a team account (as discovered
      by reading UID.info), go to Team Page.
\end{enumerate}

\subsection{Problem}

\begin{center}
{\bf Problem Page Files}
\\[1ex]
\begin{tabular}{lllll}
\TT{users/UID/PROBLEM}	                & create  & update & read & delete
\\[1ex]
\TT{users/UID/PROBLEM/PROBLEM.tex}	& upload  & -      & read & delete \\
\TT{users/UID/PROBLEM/PROBLEM.pdf}	& create  & -      & read & delete
\\[1ex]
\TT{users/UID/PROBLEM/PROBLEM.c}	& upload  & -      & read & delete \\
\TT{users/UID/PROBLEM/PROBLEM.cc}	& upload  & -      & read & delete \\
\TT{users/UID/PROBLEM/PROBLEM.java}	& upload  & -      & read & delete \\
\TT{users/UID/PROBLEM/PROBLEM.py}	& upload  & -      & read & delete \\
\TT{users/UID/PROBLEM/PROBLEM}		& create  & -      & read & delete \\
\TT{users/UID/PROBLEM/PROBLEM.class}	& create  & -      & read & delete \\
\TT{users/UID/PROBLEM/PROBLEM.pyc}	& create  & -      & read & delete
\\[1ex]
\TT{users/UID/PROBLEM/XXXX-PROBLEM.c}	& upload  & -      & read & delete \\
\TT{users/UID/PROBLEM/XXXX-PROBLEM.cc}	& upload  & -      & read & delete \\
\TT{users/UID/PROBLEM/XXXX-PROBLEM.java}& upload  & -      & read & delete \\
\TT{users/UID/PROBLEM/XXXX-PROBLEM.py}	& upload  & -      & read & delete \\
\TT{users/UID/PROBLEM/XXXX-PROBLEM}	& create  & -      & read & delete \\
\TT{users/UID/PROBLEM/XXXX-PROBLEM.class}
					& create  & -      & read & delete \\
\TT{users/UID/PROBLEM/XXXX-PROBLEM.pyc}	& create  & -      & read & delete
\\[1ex]
\TT{users/UID/PROBLEM/XXXX-PROBLEM.in}	& upload  & -      & read & delete \\
\TT{users/UID/PROBLEM/XXXX-PROBLEM.sin}	& create  & -      & read & delete \\
\TT{users/UID/PROBLEM/XXXX-PROBLEM.sout}& create  & -      & read & delete \\
\TT{users/UID/PROBLEM/XXXX-PROBLEM.fout}& create  & -      & read & delete \\
\TT{users/UID/PROBLEM/XXXX-PROBLEM.ftest}& create  & -      & read & delete \\
\TT{users/UID/PROBLEM/XXXX-PROBLEM.dout}& create  & -      & read & delete \\
\TT{users/UID/PROBLEM/XXXX-PROBLEM.score}& create  & -      & read & delete \\
\end{tabular}
\\[3ex]
{\bf Problem Page Session Data}
\\[1ex]
\begin{tabular}{lllll}
\TT{EPM\_EMAIL}	& -       & -      & read \\
\TT{EPM\_UID}	& -       & -      & read \\
\TT{EPM\_PROBLEM}
		& create  & update & read \\
\end{tabular}
\end{center}

\section{Overview}

Here we list adminstrative files and transactions, and give a
brief description of each.

\subsection{Administrative Files}

EPM uses administrative files to direct generation of files from
other files, to keep track of visibility
permissions, and for other things.  All these files are in JSON
format.  EPM does \underline{not} use any data base (like MYSQL).

Administrative files are \underline{not} uploadable by the user.
They are made by commands issued by the user to web pages, e.g.,
the \TT{PPPP.score} file is made by the user commanding
the \TT{PPPP.run} file, and the user can make their own
\TT{\ldots-PPPP.run} file using a web page.  When administrative
commands are made using a web page, the code associated with the
page checks that the file being made does not violate security.

The following administrative files the user will encounter.
Unless otherwise noted, these are visible to the user.

\begin{indpar}
\TTITEM{XXXX.mk}  File specifying how to make the file \TT{XXXX},
when this last file is generated from other files.  E.g.,
the \TT{UUUU/PPPP/00-000-PPPP.out.mk} file tells how to make
the \TT{UUUU/PPPP/00-000-PPPP.out} from 
the \TT{system/PPPP/00-000-PPPP.in},
the \TT{system\EOL /\EOL PPPP/\EOL generate-PPPP},
and \TT{UUUU/PPPP/PPPP}.

A \TT{XXXX.mk} file also tells how to make itself, so there is
no need for \TT{XXXX.mk.mk} files.  \TT{XXXX.mk} files are
made from template files.
\\[1ex]
\TTITEM{system/template/X.E->Y.F.tpl}
Template file used to make \TT{.mk} or other administrative
files.  A typical template file name ends with \TT{X.cc->X.tmpl}
and is used to make a \TT{X.mk} file that specifies how to
make a \TT{X} file from a \TT{X.cc} file, where \TT{X} is a
parameter to the template file.  In general single upper case
letters are used as parameter names for template files, and
may also appear in the names of the template files themselves.

Template files are located and used by code in web pages that
make administrative files.
\\[1ex]
\TTITEM{UUUU/credentials} Credential file for user.  Specifies
user email addresses, ip addresses, dates ip addresses last
certified and last used.
\\[1ex]
\TTITEM{UUUU/logins} Login history of user.
\\[1ex]
\TTITEM{UUUU/PPPP/uploads} Upload history of user for problem \TT{PPPP}.
Note that a user's uploads for a problem are automatically
checked into a per-user, per-problem \TT{git} database
which can be cloned by the user and partially inspected
(in particular to obtain difference listings) using EPM
web pages.
\\[1ex]
\TTITEM{UUUU/PPPP/runs} Run history of user for problem \TT{PPPP}.
\\[1ex]
\TTITEM{DDDD/user.perm} Permission control file for arbitrary
users for all files in directory \TT{DDDD} (e.g., in \TT{system},
\TT{system/PPPP}) and its subdirectories.
\\[1ex]
\TTITEM{DDDD/UUUU.perm} Permission control file for
user \TT{UUUU} and files in directory \TT{DDDD} (e.g., in directory
\TT{UUUU}).
\end{indpar}

In the above, \TT{system/} is actually a project directory.
There can be many project, and the \TT{UUUU/PPPP/UUUU.perm} file
can point to any of them, in place of \TT{system/}.

\subsection{Transactions}

The only way a user can interact with EPM is via transactions.
Each transaction is executed by entering small amounts of text
on a web page and clicking appropriately.

The common transations are:

\begin{indpar}
\BFITEM{download}  Download any file visible to the user.
\\[1ex]
\BFITEM{inspect}  Inspect various visible JSON administrative files
in a more readable format.
\\[1ex]
\BFITEM{directory}  Create or destroy problem directories and designate
a current problem directory.
\\[1ex]
\BFITEM{upload}  Upload any program source (\TT{.c}, \TT{.cc}, \TT{.java},
\TT{.py}, or \TT{.lsp}) file or any program input (\TT{.in}) file into
the current problem directory.
\\[1ex]
\BFITEM{make}  Create, edit, and destroy \TT{.mk} files, whose existence
causes the associated derived files to be made upon demand, in the
current problem directory.
\\[1ex]
\BFITEM{run}  Create, edit, and destroy \TT{.run} files, which define
runs that cause batches of derived files to be created, in the
currrent problem directory.  Execute
designated runs.
\\[1ex]
\BFITEM{permission}  Create, edit, and destroy \TT{.perm} files that control
permissions and visibility.
\\[1ex]
\BFITEM{credentials}  Inspect and remove credentials of the current user.
\\[1ex]
\BFITEM{project}  Create or destroy project directories and designate
project directories that are visible to the current problem directory.
\\[1ex]
\BFITEM{move}  Move files between the current problem directory and a
project directory.
\end{indpar}

Program source files do not have to be problem solution files.
Any program can be run so long as it opens no files and does
all its input/output via file descriptors.  Program output
can be put into \TT{.out}, \TT{.fout}, \TT{.debug}, \TT{.info}, or
\TT{.disp} files.  The last kind of file encodes X-windows commands
that can be displayed in an X-window or pdf window or placed in a
\TT{.pdf} file.  Programs an also interact with terminal windows;
for example, a program can be written to calculate combinations
(i.e., N choose K).

Problem solution programs are run without arguments, unless they
are begin debugged, in which case their output is put into \TT{.debug}
files.  Other programs can be run with or without arguments.



\end{document}
