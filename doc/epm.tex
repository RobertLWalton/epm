% Educational Problem Manager (EPM) Documentation
%
% File:         epm.tex
% Author:       Bob Walton (walton@acm.org)
% Date:		See \date below.
  
\documentclass[12pt]{article}

\usepackage[T1]{fontenc}
\usepackage{lmodern}
\usepackage{makeidx}

\makeindex

\setlength{\oddsidemargin}{0in}
\setlength{\evensidemargin}{0in}
\setlength{\textwidth}{6.5in}
\setlength{\textheight}{8.5in}
\raggedbottom

\setlength{\unitlength}{1in}

\pagestyle{headings}
\setlength{\parindent}{0.0in}
\setlength{\parskip}{1ex}

\setcounter{secnumdepth}{5}
\setcounter{tocdepth}{5}
\newcommand{\subsubsubsection}[1]{\paragraph[#1]{#1.}}
\newcommand{\subsubsubsubsection}[1]{\subparagraph[#1]{#1.}}

\newcommand{\TT}[1]{{\tt \bfseries #1}}
\newcommand{\EOL}{\penalty \exhyphenpenalty}

% Begin \tableofcontents surgery.

\newcount\AtCatcode
\AtCatcode=\catcode`@
\catcode `@=11	% @ is now a letter

\renewcommand{\contentsname}{}
\renewcommand\l@section{\@dottedtocline{1}{0.1em}{1.4em}}
\renewcommand\l@table{\@dottedtocline{1}{0.1em}{1.4em}}
\renewcommand\tableofcontents{%
    \begin{list}{}%
	     {\setlength{\itemsep}{0in}%
	      \setlength{\topsep}{0in}%
	      \setlength{\parsep}{1ex}%
	      \setlength{\labelwidth}{0in}%
	      \setlength{\baselineskip}{1.5ex}%
	      \setlength{\leftmargin}{0.8in}%
	      \setlength{\rightmargin}{0.8in}}%
    \item\@starttoc{toc}%
    \end{list}}
\renewcommand\listoftables{%
    \begin{list}{}%
	     {\setlength{\itemsep}{0in}%
	      \setlength{\topsep}{0in}%
	      \setlength{\parsep}{1ex}%
	      \setlength{\labelwidth}{0in}%
	      \setlength{\baselineskip}{1.5ex}%
	      \setlength{\leftmargin}{1.0in}%
	      \setlength{\rightmargin}{1.0in}%
	      }%
    \item\@starttoc{lot}%
    \end{list}}

\catcode `@=\AtCatcode	% @ is now restored

% End \tableofcontents surgery.

\newenvironment{indpar}[1][0.4in]%
	{\begin{list}{}%
		     {\setlength{\itemsep}{0in}%
		      \setlength{\topsep}{0in}%
		      \setlength{\parsep}{1ex}%
		      \setlength{\labelwidth}{#1}%
		      \setlength{\leftmargin}{#1}%
		      \addtolength{\leftmargin}{\labelsep}}%
	 \item}%
	{\end{list}}

\newcommand{\ITEM}{\hspace*{-0.2in}}
\newcommand{\TTITEM}[1]{\hspace*{-0.2in}{\TT{#1}}\\}
\newcommand{\BFITEM}[1]{\hspace*{-0.2in}{{\bf #1}}\\}

\begin{document}
        
\title{Educational Problem Manager}

\author{Robert L. Walton}

\date{October 31, 2019}
 
\maketitle

\newpage
\begin{center}
\large \bf Table of Contents
\end{center}

\bigskip

\tableofcontents 

\newpage

\section{Introduction}

The Educational Problem Manager permits educational programming
problems to be developed and used.  Its users merely need an
editor on their client computer: all compilation, including
text processing (e.g., latex compilation), is done on the
server.  The interface is a web browser.

The main features are:

\begin{itemize}
\item Server Operating System: linux (CentOS or Ubuntu)
\item Documentation Language: latex (pdflatex)
\item System Programming Language: php (version 5.6 or later)
\item User Programming Languages:
\begin{itemize}
\item C (gcc)
\item C++ (g++)
\item JAVA (OpenJDK)
\item COMMONLISP (SBCL)
\end{itemize}

\end{itemize}

This document is a comprehensive manual for both users and
system maintainers.  There are also HTML web pages to guide
users through common user actions.

\section{Overview}

EPM in effect is a file system, where some files are source
files uploaded by users, and many files are automatically
generated from source files or from other generated files.
The typical user does \underline{not} have control over
the precise commands use to generate files from other
files, but can merely upload source files and cause generated
files to be generated.  The user can also display files
and information derived from files, and download source
and generated files.

\subsection{Problem Files}

So let us look at a detailed example involving a user named
\TT{UUUU} and a problem named \TT{PPPP}.

There is a project named `\TT{system}' which all users can access.
Suppose this contains the problem \TT{PPPP}.  This means the
following files exist:

\begin{indpar}
\TTITEM{system/PPPP/PPPP.tex}  Problem description in latex.
    Not visible to the user.
\\[1ex]
\TTITEM{system/PPPP/PPPP.pdf}  Problem description made from
    \TT{PPPP.tex} by using \TT{pdflatex}.
    Visible to user.
\\[1ex]
\TTITEM{system/PPPP/supplement-PPPP.tex}  Problem supplementary
    documentation in latex.  Documents generate and filter programs:
    see below.  Not visible to the user.
\\[1ex]
\TTITEM{system/PPPP/supplement-PPPP.pdf}  Problem supplementary
    documentation made from
    \TT{PPPP.tex} by using \TT{pdflatex}.
    Visible to user.
\\[1ex]
\ITEM\begin{tabular}[t]{@{}l}
     \TT{system/PPPP/00-000-PPPP.in} \\
     \TT{system/PPPP/00-001-PPPP.in} \\
     \ldots\ldots\ldots\ldots\ldots \\
     \TT{system/PPPP/01-000-PPPP.in} \\
     \TT{system/PPPP/01-001-PPPP.in} \\
     \ldots\ldots\ldots\ldots\ldots \\
     \end{tabular}
     ~~~~
     \begin{tabular}[t]{p{3in}}
     Test case input files.  Usable in runs,
     but not visible to user except for \TT{00-\ldots.in}
     files that are visible sample inputs.
     \end{tabular}
\\[1ex]
\TTITEM{system/PPPP/generate\_PPPP.cc}  Source code for generate
    program that generates actual test case input from \TT{.in} files.
    Not visible to user.
\\[1ex]
\TTITEM{system/PPPP/generate\_PPPP}  Binary of generate
    program made by using \TT{g++} on \TT{gen\-erate\_\EOL PPPP.cc}.
    Visible to user.
\\[1ex]
\TTITEM{system/PPPP/PPPP.cc}  Judge's solution to the problem.
    Accepts actual test case input from the generate program,
    and produces actual test case output (\TT{.test} files below).
    Not visible to user.
\\[1ex]
\TTITEM{system/PPPP/PPPP}  Binary of judge's solution to the problem,
    made from \TT{PPPP.cc} by using \TT{g++}.
    Not visible to user, but can be used during a run to regenerate
    other \TT{system/PPPP} files.
\\[1ex]
\ITEM\begin{tabular}[t]{@{}l}
     \TT{system/PPPP/00-000-PPPP.test} \\
     \TT{system/PPPP/00-001-PPPP.test} \\
     \ldots\ldots\ldots\ldots\ldots \\
     \TT{system/PPPP/01-000-PPPP.test} \\
     \TT{system/PPPP/01-001-PPPP.test} \\
     \ldots\ldots\ldots\ldots\ldots \\
     \end{tabular}
     ~~~~
     \begin{tabular}[t]{p{3in}}
     Judge's test case output files.  Each is made by running
     the corresponding \TT{.in} file first through the generate
     program and then through the judge's solution program.
     Usable in runs but not visible to user except for \TT{00-\ldots.test}
     files that are visible sample judge's outputs.
     \end{tabular}
\\[1ex]
\TTITEM{system/PPPP/PPPP.run}  Standard `run' file which lists
    the test cases to be run in the order they are to be
    run.  For example:
   `{\TT{00-000-PPPP}, \TT{00-001-PPPP}, \ldots}'.
    Visible to the user.
\\[1ex]
\TTITEM{system/PPPP/sample-PPPP.run}  Ditto but only includes sample tests.
\end{indpar}

All compiled programs follow the convention that they cannot
open files, but must use file descriptors, in particular
the standard input and standard output.  If a solution writes
the standard error, it is deemed to have crashed (because
runtime systems write the standard error if they detect
JAVA errors, memory faults, etc.).  Sometimes a non-solution
will read from file descriptor 3, which is used as a secondary
input when two inputs are required, as by a generate program
(or a filter program, described below).

All this is to allow all compiled programs to run in a sandbox,
so as to protect the server from malicious programs, including,
for example, generate programs submitted by users developing
their own problems.

The user directory has the following files, all of which are
visible to the user unless noted otherwise:

\begin{indpar}
\TTITEM{UUUU/PPPP/PPPP.py}  User's solution to the problem.
    May be in a different programming language from the judge's
    solution (e.g., \TT{.py} is python, \TT{.cc} is C++).
    Uploaded by the user (and thus modifiable by the user).
\\[1ex]
\TTITEM{UUUU/PPPP/PPPP}  Binary of users's solution to the problem,
    typically a \TT{bash} script that executes `\TT{python3 PPPP.py}'
    (if the user's solution were C++, \TT{g++} would be used to
    compile \TT{PPPP.cc}).
\\[1ex]
\ITEM\begin{tabular}[t]{@{}l}
     \TT{UUUU/PPPP/00-000-PPPP.out} \\
     \TT{UUUU/PPPP/00-001-PPPP.out} \\
     \ldots\ldots\ldots\ldots\ldots \\
     \TT{UUUU/PPPP/01-000-PPPP.out} \\
     \TT{UUUU/PPPP/01-001-PPPP.out} \\
     \ldots\ldots\ldots\ldots\ldots \\
     \end{tabular}
     ~~~~
     \begin{tabular}[t]{p{3in}}
     User's test case output files.  Each is made by running
     the corresponding \TT{.in} file first through the generate
     program and then through the user's solution program.
     Only sample test case files and the file for the first
     failed test case are visible to the user.
     \end{tabular}
\\[1ex]
\TTITEM{UUUU/PPPP/PPPP.score}  Score associated with
    the `\TT{system/PPPP/PPPP.run}' file.  Either says the run was completely
    correct, or specifies the first test case that failed.
    Also contains a history of past scores.
    Total scores are computed using the number of failed runs before
    the first successful run.
\\[1ex]
\TTITEM{UUUU/PPPP/sample-PPPP.score}  Ditto but only include sample tests
in \TT{system/\EOL PPPP/\EOL sample-\EOL PPPP.run}.
\\[1ex]
\ITEM\begin{tabular}[t]{@{}l}
     \TT{system/PPPP/00-000-PPPP.gin} \\
     \TT{system/PPPP/00-001-PPPP.gin} \\
     \ldots\ldots\ldots\ldots\ldots \\
     \TT{system/PPPP/01-000-PPPP.gin} \\
     \TT{system/PPPP/01-001-PPPP.gin} \\
     \ldots\ldots\ldots\ldots\ldots \\
     \end{tabular}
     ~~~~
     \begin{tabular}[t]{p{3in}}
     Result of running corresponding \TT{.in} files through the 
     generate program.  These files represent the input actually
     seen by the user's \TT{PPPP} solution program.  Not all
     these files are computed: a \TT{.gin} file may be computed
     only if the corresponding \TT{.in} file is visible to the
     user.
     \end{tabular}

\end{indpar}

The user, to develop a problem solution, repeatedly uploads and
the \TT{PPPP.py} file and executes a \TT{.run} file.
Each test produces a \TT{.in},
\TT{.gin}, \TT{.out}, and \TT{.test} file that can be inspected,
along with differences between the \TT{.out} and \TT{.test}
files.

A general principal is that the user is only allowed to look
at input and output files (\TT{.in}, \TT{.gin}, \TT{.test}, \TT{.out} files)
of sample test cases (\TT{00-\ldots} files), and of the `first
failed test case'.  This last is computed by executing a run
using a \TT{.run} file, and is the first test case listed in
that file on which the solution fails.

In the above we have assumed that the problem is such that the
\TT{.out} file is uniquely determined by the \TT{.in} file,
with perhaps minor differences such as spacing and
the exact number of decimal places printed (as long as number
agree within the tolerance specified by the problem).
However some problems have many possible correct solutions:
for example a maze problem asking for a shortest path in which
there may be several shortest pathes.
For these the following files are added.

In the \TT{system/PPPP} directory:

\begin{indpar}
\TTITEM{system/PPPP/filter\_PPPP.cc}  Source code for filter
    program that accepts actual test case input from the generate
    program and actual test case output from a problem solution
    and produces filtered output (\TT{.ftest} or \TT{.fout} files,
    see below).
    Not visible to the user.
\\[1ex]
\TTITEM{system/PPPP/filter\_PPPP}  Binary of filter
    program made by using \TT{g++} on \TT{fil\-ter\_\EOL PPPP.cc}.
    Visible to the user.
\\[1ex]
\ITEM\begin{tabular}[t]{@{}l}
     \TT{system/PPPP/00-000-PPPP.ftest} \\
     \TT{system/PPPP/00-001-PPPP.ftest} \\
     \ldots\ldots\ldots\ldots\ldots \\
     \TT{system/PPPP/01-000-PPPP.ftest} \\
     \TT{system/PPPP/01-001-PPPP.ftest} \\
     \ldots\ldots\ldots\ldots\ldots \\
     \end{tabular}
     ~~~~
     \begin{tabular}[t]{p{3in}}
     Judge's test case filtered output files.  Each is made by running
     the corresponding \TT{.in} file first through the generate
     program, then through the judge's solution program, and lastly
     through the filter program.
     Useable in runs but not visible to user,
     except for \TT{00-\ldots.ftest} files that are visible sample outputs.
     \end{tabular}
\end{indpar}

For test cases on which a solution is successful, the filtered output
typically consists of just one line containing either \TT{OK}, or in
the case of a problem that tries to minimize some number while providing
extra information, just the number.  If the solution is unsuccessful,
the filtered output will have error messages.  For example, if a shortest path
through a maze is to be output, and the \TT{.test} file has the path,
the filtered output will have the length of the path if the path is
legal, and error messages otherwise.

When there is a filter program, the \TT{UUUU/PPPP} directory has:

\begin{indpar}
\ITEM\begin{tabular}[t]{@{}l}
     \TT{system/PPPP/00-000-PPPP.fout} \\
     \TT{system/PPPP/00-001-PPPP.fout} \\
     \ldots\ldots\ldots\ldots\ldots \\
     \TT{system/PPPP/01-000-PPPP.fout} \\
     \TT{system/PPPP/01-001-PPPP.fout} \\
     \ldots\ldots\ldots\ldots\ldots \\
     \end{tabular}
     ~~~~
     \begin{tabular}[t]{p{3in}}
     User's test case filtered output files.  Each is made by running
     the corresponding \TT{.in} file first through the generate
     program, then through the user's solution program, and lastly
     through the filter program.
     Not visible to user except sample test cases and the first failed
     test case.
     \end{tabular}
\end{indpar}

A user may develop a problem.  In this case the \TT{UUUU/PPPP} problem
will have all the files that were in the \TT{system/PPPP} directory
above, all these files will be visible to the user, and the user
will be able to upload all the \TT{.tex}, \TT{.cc}, \TT{.py} etc
files (but not the administrative files in the next section).

\subsection{Administrative Files}

EPM uses administrative files to direct generation of files from
other files, to keep track of visibility
permissions, and for other things.  All these files are in JSON
format.  EPM does \underline{not} use any data base (like MYSQL).

Administrative files are \underline{not} uploadable by the user.
They are made by commands issued by the user to web pages, e.g.,
the \TT{PPPP.score} file is made by the user commanding
the \TT{PPPP.run} file, and the user can make their own
\TT{\ldots-PPPP.run} file using a web page.  When administrative
commands are made using a web page, the code associated with the
page checks that the file being made does not violate security.

The following administrative files the user will encounter.
Unless otherwise noted, these are visible to the user.

\begin{indpar}
\TTITEM{XXXX.mk}  File specifying how to make the file \TT{XXXX},
when this last file is generated from other files.  E.g.,
the \TT{UUUU/PPPP/00-000-PPPP.out.mk} file tells how to make
the \TT{UUUU/PPPP/00-000-PPPP.out} from 
the \TT{system/PPPP/00-000-PPPP.in},
the \TT{system\EOL /\EOL PPPP/\EOL generate-PPPP},
and \TT{UUUU/PPPP/PPPP}.

A \TT{XXXX.mk} file also tells how to make itself, so there is
no need for \TT{XXXX.mk.mk} files.  \TT{XXXX.mk} files are
made from template files.
\\[1ex]
\TTITEM{system/template/X.E->Y.F.tpl}
Template file used to make \TT{.mk} or other administrative
files.  A typical template file name ends with \TT{X.cc->X.tmpl}
and is used to make a \TT{X.mk} file that specifies how to
make a \TT{X} file from a \TT{X.cc} file, where \TT{X} is a
parameter to the template file.  In general single upper case
letters are used as parameter names for template files, and
may also appear in the names of the template files themselves.

Template files are located and used by code in web pages that
make administrative files.
\\[1ex]
\TTITEM{UUUU/credentials} Credential file for user.  Specifies
user email addresses, ip addresses, dates ip addresses last
certified and last used.
\\[1ex]
\TTITEM{UUUU/logins} Login history of user.
\\[1ex]
\TTITEM{UUUU/PPPP/uploads} Upload history of user for problem \TT{PPPP}.
Note that a user's uploads for a problem are automatically
checked into a per-user, per-problem \TT{git} database
which can be cloned by the user and partially inspected
(in particular to obtain difference listings) using EPM
web pages.
\\[1ex]
\TTITEM{UUUU/PPPP/runs} Run history of user for problem \TT{PPPP}.
\\[1ex]
\TTITEM{DDDD/user.perm} Permission control file for arbitrary
users for all files in directory \TT{DDDD} (e.g., in \TT{system},
\TT{system/PPPP}) and its subdirectories.
\\[1ex]
\TTITEM{DDDD/UUUU.perm} Permission control file for
user \TT{UUUU} and files in directory \TT{DDDD} (e.g., in directory
\TT{UUUU}).
\end{indpar}

In the above, \TT{system/} is actually a project directory.
There can be many project, and the \TT{UUUU/PPPP/UUUU.perm} file
can point to any of them, in place of \TT{system/}.

\subsubsection{Transactions}

The only way a user can interact with EPM is via transactions.
Each transaction is executed by entering small amounts of text
on a web page and clicking appropriately.

The common transations are:

\begin{indpar}
\BFITEM{download}  Download any file visible to the user.
\\[1ex]
\BFITEM{inspect}  Inspect various visible JSON administrative files
in a more readable format.
\\[1ex]
\BFITEM{directory}  Create or destroy problem directories and designate
a current problem directory.
\\[1ex]
\BFITEM{upload}  Upload any program source (\TT{.c}, \TT{.cc}, \TT{.java},
\TT{.py}, or \TT{.lsp}) file or any program input (\TT{.in}) file into
the current problem directory.
\\[1ex]
\BFITEM{make}  Create, edit, and destroy \TT{.mk} files, whose existence
causes the associated derived files to be made upon demand, in the
current problem directory.
\\[1ex]
\BFITEM{run}  Create, edit, and destroy \TT{.run} files, which define
runs that cause batches of derived files to be created, in the
currrent problem directory.  Execute
designated runs.
\\[1ex]
\BFITEM{permission}  Create, edit, and destroy \TT{.perm} files that control
permissions and visibility.
\\[1ex]
\BFITEM{credentials}  Inspect and remove credentials of the current user.
\\[1ex]
\BFITEM{project}  Create or destroy project directories and designate
project directories that are visible to the current problem directory.
\\[1ex]
\BFITEM{move}  Move files between the current problem directory and a
project directory.
\end{indpar}

Program source files do not have to be problem solution files.
Any program can be run so long as it opens no files and does
all its input/output via file descriptors.  Program output
can be put into \TT{.out}, \TT{.fout}, \TT{.debug}, \TT{.info}, or
\TT{.disp} files.  The last kind of file encodes X-windows commands
that can be displayed in an X-window or pdf window or placed in a
\TT{.pdf} file.  Programs an also interact with terminal windows;
for example, a program can be written to calculate combinations
(i.e., N choose K).

Problem solution programs are run without arguments, unless they
are begin debugged, in which case their output is put into \TT{.debug}
files.  Other programs can be run with or without arguments.



\end{document}
